%%% notes.tex --- Some notes relative to SamSpecCoEN
%%% Author: cazencott@falstaff

\documentclass[12pt,a4paper]{article}
\usepackage[T1]{fontenc}     
\usepackage[utf8]{inputenc}            

\usepackage[runin]{abstract}           
\usepackage{amssymb}                   
\usepackage{amsmath}                   
\usepackage{amsthm}                    
\usepackage{enumitem}                  
\usepackage[margin=2cm]{geometry}      
\usepackage{graphicx}                  
\usepackage{hyperref}                  
\usepackage[svgnames]{xcolor}          

% Hyperref
% for a list of colors: p38 of http://mirrors.ibiblio.org/CTAN/macros/latex/contrib/xcolor/xcolor.pdf
\hypersetup{colorlinks=true,    % false: boxed links; true: colored links
            linkcolor=teal,     % color of internal links
            citecolor=teal}     % color of links to bibliography
% Font
\usepackage[charter]{mathdesign}

% Edits
\newcommand{\caz}[1]{{\color{purple}[TODO (CA) : #1]}}


\title{Sample-specific co-expression networks}
\author{Chloé-Agathe Azencott}
\date{April 2016}

\begin{document}

\maketitle

\section{Stability}
Others have observed that although incorporating biological networks in the analysis of the ACES data~\cite{staiger2013} does not necessarily improve performance, it helps discovering more {\em stable} predictors~\cite{allahyar2015}.

\section{FERAL}
FERAL~\cite{allahyar2015} is a state-of-the-art approach for network-based breast cancer outcome prediction from gene-expression profiles.\\

Let us assume $n$ samples, $p$ genes, the expression $X \in \mathbb{R}^{n \times p}$ of these $p$ genes in the $n$ samples, the outcome $Y \in \{0, 1\}^n$ for the $n$ samples, and a biological network over the $p$ genes.\\

FERAL proceeds as follows:
\begin{enumerate}
\item The biological network is used to define $G=p$ gene sets $\Psi_g$ as groups of
  $k=10$ genes, formed by a seed genes (each of the $p$ genes is a seed gene) 
  and $k-1$ of its closest neighbors (randomly chosen in case of ties).  
\item The expression data is $z$-score normalized (i.e. mean $0$, standard deviation $1$).
\item Meta-features are computed for each gene set by 
  aggregating the expression levels of the genes that compose it
  according to the following $m=6$ operators:
  \begin{itemize}
  \item {\em Average} operator:  \[\mbox{Average}(\Psi_g) = \frac{1}{k} \sum_{j \in \Psi_g} X_j\]
  \item {\em Direction Aware Average} operator: 
    \[
    \mbox{DAA}(\Psi_g) = \frac{1}{k} \sum_{j \in \Psi_g}  \mbox{sgn}(\rho_j) X_j,
    \]
    where $\rho_j$ denotes the correlation between $X_j$ and the class label $Y$.
  \item On the same model, the {\em median}, {\em variance}, {\em min} and {\em max} operators.
  \end{itemize}

  The initial data representation $X$ is augmented by the resulting $m \times G$ features, 
  resulting in a final representation $\tilde X \in \mathbb{R}^d$ with $d=(m+1) \times p$.

\item A Sparse Group Lasso classifier~\cite{friedman2010} is trained:
  $G$ groups of features are formed, corresponding to the $G$ gene sets. 
  Each group contains the expression levels of the $k$ genes of the gene set, 
  as well as the $m$ meta-features computed for this gene set. 

  \[
  \min_{\beta \in \mathbb{R}^d} \left( l(Y, \tilde X  \beta) + \lambda_1 || \beta||_1 + 
  \lambda_2 \sum_{g=1}^G \sqrt{p_g}\, ||\beta_g||_2 \right),
  \]
  where $p_g = |\Psi_g|+m = k+m$ is the size of group $g$, 
  $\beta_g$ is the weight vector $\beta$ restricted to the indices 
  corresponding to the features in group  $g$, and $l$ is the logistic loss function.

  \begin{itemize}
  \item Sample weighting is added to mitigate the effects of class imbalance.
  \item The parameters $\lambda_1$ and $\lambda_2$ are chosen by inner cross-validation.
  \end{itemize}
\item Each gene is assigned a score that is the average of the weights assigned to it and to any group that it belongs to:
  \[
  score(j) = \frac{1}{|g: j \in \Psi_g|} \sum_{g: j \in \Psi_g} \beta_g
  \]
\end{enumerate}


\section{LIONESS}
LIONESS~\cite{kuijjer2015} is, to the best of our knowledge, the only existing method to build sample-specific co-expression networks from gene expression data.\\

LIONESS considers that the global network is made of the aggregation of individual contributions from each sample. 
Denoting by $e_{ij}^\alpha$ the weight of the edge between $i$ and $j$ computed over all samples;
by $e_{ij}^{(\alpha - q)}$ the weight of the edge between $i$ and $j$ computed over all samples but sample $q$;
and by $e_{ij}^q$ the weight of the edge between $i$ and $j$ contributed by sample $q$:
    \[
    e_{ij}^\alpha = \sum_{s=1}^n \frac{1}{n} e_{ij}^s 
    \mbox{ and }
    e_{ij}^{(\alpha - q)} = \sum_{s=1}^n \frac{1}{(n-1)} e_{ij}^s
    \]

Hence
    \[
    e_{ij}^q = \frac{1}{n} \left(  e_{ij}^{\alpha} - e_{ij}^{(\alpha - q)} \right) + e_{ij}^{(\alpha - q)}.
    \]

In the case of a network computed using Pearson's correlation, assuming the number $n$ of samples to be large,~\cite{kuijjer2015} show that the individual contributions of each sample can be approximated by
    \[
    e_{ij}^q \approx \frac{n}{n-1} \left( \frac{X_i^q - \bar {X_i}}{S_{X_i}} \right) 
    \left( \frac{X_j^q - \bar {X_j}}{S_{X_j}} \right)
    \]
    where $S_{X_i} = \sqrt{\frac{1}{n-1} \sum_{s=1}^n \left( X_i^s - \bar{X_i} \right)^2}$ is the sample standard deviation of gene $i$.


\section{Reglines}
Also based on Pearson's correlation, we propose to compute sample-specific edge weights based not on the contribution of each sample to the total correlation, but on the deviation of each sample from the overall relationship between both genes. To this aim, we use the distance of the sample to the regression line between both variables.

Given a pair $(i, j)$ of genes, We fit a regression line $\beta_0^{ij} + \beta_1^{ij} x_i = x_j$, and the edge weight of $(i, j)$ for sample $q$ is given by:

\[
e_{ij}^q = \frac{|\beta_1^{ij} x_i - x_j + \beta_0^{ij}|}{\sqrt{(\beta_1^{ij})^2+1}}.
\]

\bibliographystyle{plain}

\bibliography{references}

\end{document}
